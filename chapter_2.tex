%   Filename    : chapter_2.tex 
\chapter{Review of Related Literature}
\label{sec:relatedlit}

\section{Document Tracking Systems}

\subsection{Logistic Tracking Systems}
Logistics tracking refers to the systems and ways of tracking resources during their movement and storage. With logistics tracking, the sender, receiver, and moderator will know where the resources are physically located at any time and schedule delivery. Tracking systems usually use waybills, a document attached to the resources that specify at minimum their nature, point of origin, and destination.\cite{waybill} As millions of cargos are transported every day, waybills in these systems make it possible to track the progress of each delivery. 

According to \citeA{airspeed_2020}, a tracking system works with the following sequence:
\begin{enumerate}
    \setlength\itemsep{1em}
    \item Bar Code Generation - a unique ID is assigned to each resource that contains the important details of the package, such as destination, name of sender and receiver, address, and crucial information needed by the buyer. Once generated, bar code generated or waybill is attached to the resources
    \item Scan and Track - allows data attached to the waybill to be received by the courier service and updates the location in the database.
    \item Tracking Progress - real-time tracking becomes available to both sender and receiver as well the moderators of the logistics through a dedicated page.
    \item Product Delivery - the resources have reached the destination and are received by the recipient and, indicating that the delivery is successful.
\end{enumerate}

Currently, e-commerce web applications utilize both in-app tracking and logistics waybill tracking \cite{studio_brand, amadora_2021}. This implementation makes it more convenient for customers to track their parcels at a glance on the e-commerce application, and in-depth details on the logistics tracking page. 

ReTrac will follow a similar concept such as the generation of a unique waybill ID from the submission of the online form and will end the tracking when the request confirms the receipt of the document. The system\textsc{\char13}s real-time update will help the requester monitor the document request and as well provide adequate information in the case issues arise on that certain transaction. 

\subsection{Document Tracking Systems in Higher Education Institutions}

Several document tracking systems are implemented across academic institutions, government agencies, and private institutions. However, traditional recording systems such as logbooks and pen logging are still being used as safeguard databases. The University of the Philippines (UP)  System has implemented a centralized system through the UP Information Technology Development Center called UP Document Tracking System.  This system is capable of tracking document trails within offices. It includes the origin office, receiving office, and personnel handling the documents. \cite{UP_DTS} 

A similar model is tested in the Schools Division of Parañaque City \cite{emralino} where they created a system that helps the inter-offices to manage the document trail. However, tickets as waybills are used upon the request of the user for faster compliance in line with the Republic Act No. 9485 \cite{ra945}. The developers of the said system developed it through brainstorming, interviewing different users of the document trail, and organized orientation and training on the utilization of DoTS. 

In \citeyearNP{dts_hei}, \citeauthor{dts_hei} proposed a document tracking system for the utilization of Philippine Higher Education Institutions. Tarlac Agricultural University became the sample local of the study where user requirements specifications, design and implementation, validation, and evaluation became part of the software development process. The system is evaluated by forty (40) office personnel and five (5) experts for the user interface and functionality, database design, and security. Like the previous study above, bureaucracy and compliance to the RA 9465 or the Anti Red-tape became one of the key factors in the development of the system. 

As we adapt to paperless transactions, information systems, specifically document tracking and tracing are necessary to have an integrated, interconnected, and interpolated communication between offices and the requester \cite{felipe_mendez_2020}. 

DocuTrak was implemented in the year 2003 in Diliman Network (DILNET) as part of the university\textsc{\char13}s initiated computerized projects such as Computer Registration System (CRS), Student Records System (SRS), Faculty Information System (FIS), Socialized Tuition and Financial Assistance Program (STFAP) Online, the Integrated Library System (iLib), and University Virtual Learning Environment (UVLE). As assessed by \citeA{sueno}, in-house developed DTS emphasizes the compliance of the system with the International Organization for Standardizations (ISO) definition of the records system. DocuTrak can identify and monitor bottlenecks in the document trail. Three years after the initial implementation in UP Diliman, the University of the Philippines Visayas Data and Information System Program (DISP) Office requested the same system to implement on its campuses. Unlike CRSIS, DocuTrak did not achieve much popularity and full-scale support. Unlike other records and information systems in UP Visayas and UP Diliman, DocuTrack lacks sanctions for non-use and non-users which makes the system optional since most of its features are readily available on manual DTS  implemented under Memorandum Circular No. 13 back 1976. Recently, UP Visayas adopted once again the system-wide DTS called UP Document Routing System (UP DRS) at the same time maintains the usage of manual DTS \cite{camposano_2020}. Unlike its predecessor, ReTrac will have integration and would be accessible not only to the currently employed staff and students but also to the alumni. This will also integrate UP mail domains for authentication and payment options, thus not only as a document tracking but also as a management tool. 


As tracking and tracing resources have been widespread in industries as such e-commerce, logistics, and shipment, having essential information about the status and location of resources allows better planning, scheduling, and monitoring. Implementing some form of tracking system enables monitoring of performance and behaviors of concerned offices in document processing. However, privacy issues are predominant in offices, sender, and receiver such that their essential information and work behaviors are accessible to certain roles, this perceived risk must be considered when implementing a DTS. \cite{perception_tracking}. In the process of developing ReTrac, the developers consider DocuTrak and UP DRS as their predecessors. The existing \citeA{up_privacy_personnel} for the Cash Office and Office of the College Secretary, \citeA{UP_Privacy_stud} will be observed to safeguard the privacy and data of its employees, staff, students, and alumni.

\section{Electronic Payment}
According to \citeA{yu_hsi_kuo_2002}, the worldwide proliferation of the Internet gave birth to electronic commerce, a type of business that allows electronic transfer of transactions. Electronic commerce grew because of the openness, speed, anonymity, digitization, and global accessibility characteristics of the Internet. It facilitated real-time business activities which included advertising, querying, sourcing, negotiation, auction, ordering, and paying for merchandise. 

The level of security in each step of the transaction is the main concern with electronic payment because money and merchandise are transferred with no direct contact between parties involved in the transactions. If there is even a slight chance that the payment system may not be secure, trust and confidence in the system will begin to erode, destroying the infrastructure needed for electronic commerce.

As stated in \citeA{yu_hsi_kuo_2002}, there are four categories of electronic payment systems: (1) Online credit card payment, (2) Electronic cash, (3) Electronic checks, and (4) Small payments. Each of these systems has its advantages and disadvantages:

\begin{description}
   \item $\bullet$ \textbf{Online credit card payment} - the most popular system widely used by many establishments because of its acceptability all around the world and it is a relatively safe mode of payment. The disadvantage of using this system is that using credit cards have high transaction fees and have a limit on how much you can charge the credit card
   \item $\bullet$ \textbf{Electronic cash} - the second most popular mode of payment because of its accessibility, low transaction fees, the payment between parties, and anonymity. The disadvantage of using electronic cash is that electronic cash is not replaceable once it is lost.
   \item $\bullet$ \textbf{Electronic checks} - are mostly used by government and private institutions because transactions made between institutions usually involve large amounts of money and electronic checks make transferring large amounts of money possible because they do not have a limit on how much money is transferred at a time. The disadvantage of using electronic checks is that the cost of using electronic checks is high. They can only be used in the virtual world, and they do not protect a user \textsc{\char13}s privacy.
   \item $\bullet$ \textbf{Small payments} - mostly used by establishments that provide certain services. Here, a consumer can only pay for a certain amount for a certain service that the establishment provides instead of paying for a membership for all the services. This makes it convenient for consumers that are not frequent users. One disadvantage of using this mode of payment is that they are not brought forth by financial organizations and they do not use traditional financial systems or methods as their structure. 
\end{description}

\subsection{Online Banking and Mobile Banking}
An article by \citeA{zoleta_2021} defines online banking as any transaction involving finances that are performed over the internet through a bank\textsc{\char13}s website on a user\textsc{\char13}s computer. For users to use online banking services of the bank that they have accounts in, the banks usually require them to register for an online account whenever a user opens a new account or if they have existing accounts, they inform the users to register for one. Once users are registered, they can complete bank transactions (i.e. fund transfer, bills payment) anytime and anywhere without going to a physical branch of a bank.
	
Mobile banking is essentially the same as online banking. It also enables users to complete bank transactions online anytime and anywhere. The only difference is how these two services are accessed by users. Online banking can be accessed through computers and mobile banking can be accessed through mobile phones. 
	
ReTrac considered using online and mobile banking as one of the payment options. This will enable students and alumni of the University of the Philippines who have access to these kinds of services to pay for their documents.

\subsection{E-wallets}
In an article by \citeA{pobre_2021}, GCash is one of the popular e-wallets used in the Philippines. Through GCash, users can top-up load, send and receive money to other users, pay bills, and transfer money to bank accounts partnered with GCash. Users can also scan QR codes generated within the GCash application for faster transactions.

Another popular e-wallet in the Philippines is Paymaya. Paymaya is very similar to GCash. Users can also top-up load, send and receive money to other users, pay bills, and transfer money to bank accounts partnered with Paymaya. Like GCash, users can also generate QR codes within the application for faster transactions. 

ReTrac will be using Paymaya as one of the payment options. This will enable students and alumni of the University of the Philippines Visayas, especially those who do not have bank accounts to pay for the documents that they requested. 

\subsection{Contactless Payment}
According to \citeA{qr_code}, QR codes can be used to store bank account and credit card information. They can also be designed specifically to work with particular payment options. With that in mind, some e-wallets (i.e. GCash and Paymaya) use QR codes to make payment for requesting documents faster. 

ReTrac will focus on contactless payment and utilization of banking portals as its primary payment method upon the availability of the said system with UPV Cash Office. With the shift to online transactions over the pandemic \cite{felipe_mendez_2020}, payment options other than cash should be available to the requester in a contactless way and can be processed remotely. PayMaya utilizes QR scanning to pay feature and Landbank uses Link.BizPortal.



