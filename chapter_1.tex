%   Filename    : chapter_1.tex 
\chapter{Introduction}
\label{sec:researchdesc}    %labels help you reference sections of your document

\section{Overview of the Current State of Technology}
\label{sec:overview}

The document request in the University of the Philippines Visayas (UPV) can be done in two ways, (1) through requesting in-person, and (2) requesting through email. Both approaches consist of downloading and filling out the request form, paying the document cost to the Cash Office, then submitting the receipt to the respective office that handles the document to be requested. After that, the requester would wait for the document to be ready which takes at least one week or longer on occasions such as the enrollment period.

Additionally, there is an existing online portal for document tracking that can be used by UP faculty and staff across the UP System. The portal, named UP Document Routing System (UP DRS), sets a tracking number for each document which is used to create a document trail from the office it originated to every office that handles it.

However, when the student request documents in-person or via email, it cannot be tracked or determined if the request was processed. The UP DRS, on the other hand, has a tracking feature, but it is only available to the faculty and staff. This study intends to close that gap by developing a platform for requesting and tracking documents.

\section{Problem Statement}
The process in the current document requesting system used by the University of the Philippines Visayas is very inefficient. Students and alumni often encounter problems with the current system among which are discussed below.

When requesting documents through email, the requester has to open the website provided by each office from which the PDF form can be accessed, then the requester would download and fill out the form before sending it to the respective office that handles the document. After sending the form, the requester has to pay the document cost to the Cash Office, which is a separate office from the one who handles the document requests, through Land Bank and GCash. After sending the required amount, the requester has to send an email to the Cash Office regarding the payment transaction. When the Cash Office confirms the payment, they will send a scanned copy of the receipt to the requester. Upon receiving the receipt, the requester has to forward it to the office that handles the document request, for the request to be processed.

There are instances where the office overlooked some requests due to high amounts of transactions flooding the email that is used for document requests, especially during the enrollment period, and on some occasions, requesters need to pass signed documents. Requesters may also send document requests to the wrong office which adds to the delay in the processing time.

There are also difficulties for both Cash Office and requesters during the payment process since the Cash Office only caters to money transfers through Land Bank and GCash. Some students don't have Land Bank and GCash accounts so they tend to use the accounts of their friends and relatives which makes it more difficult to track who sent the payment. Moreover, payment transactions take up to 24 hours to update in the account of the Cash Office, thus delaying the sending of receipts to the requester.

Furthermore, the decentralized nature of the document request system in the University of the Philippines makes it more difficult to track the process status of the document request. Although the UP Document Routing System is usable for UP faculty and staff, it can only track documents and not create document requests to the concerned offices.


\section{Research Objectives}
\label{sec:researchobjectives}

\subsection{General Objective}
\label{sec:generalobjective}

The study aims to develop a centralized web application that will serve as a portal for students and alumni to request documents and track their requests, as well as, for UP offices to process the said requests. This web application will be called ReTrac.


\subsection{Specific Objectives}
\label{sec:specificobjectives}


The specific objectives of ReTrac are the following:
\begin{itemize}
    \item To improve the accessibility of requesting documents from offices inside the university
    \item To improve the efficiency of processing documents inside the university
    \item To make a centralized system for requesting and processing documents inside the university.
    \item Students and alumni can request documents, and pay through integrated payment methods through the app, and offices can see and process the said request and provide status updates.
    \vfill
\end{itemize}


\section{Scope and Limitations of the Research}
\label{sec:scopelimitations}

This study focuses on requesting, tracking, and paying for certain documents that students and alumni requests to the University of the Philippines Visayas (UPV) College of Arts and Sciences (CAS). These documents may be a copy of grades, transcripts, certificate of good moral character, certificate of enrollment, certificate of non-contact/conforme, certificate of units earned, certificate of year level standing, certificate of GWA, photocopy of form 5, and/or other documents students and alumni may want to request from the Office of the College Secretary. Payment for these documents is through the University of the Philippines Visayas’ Cash Office. Requesting, paying, and tracking the documents will be through ReTrac.

For now, this study is limited only to the College of Arts and Sciences Office of the College Secretary (ColSec), the University of the Philippines Visayas’ Cash Office, and the students and alumni of the College of Arts and Sciences. Once this study is successful, the document tracking system will later be adapted to all the other colleges and offices inside the University of the Philippines Visayas. 



\section{Significance of the Research}
\label{sec:significance}

This study aims to solve the inefficiencies of the current system of requesting documents inside the University of the Philippines Visayas. Once ReTrac is successful and integrated into the university, it should improve the ease of requesting and paying for certain documents from certain offices inside the university for students and alumni. The system will also enable students and alumni to keep track of the current status of the documents that they requested. The system will also lessen the burden of staff as they would not have to worry about lost requests. And since the requesters can track their requests, the staff need not respond to every email asking for the request status.

